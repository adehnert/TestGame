\documentclass[green]{testgame}
\def\gamename{Gamewriting}
\def\gamedate{Whenever}

\begin{document}

\newcommand{\ter}[1]{\fbox{\parbox{6.5in}{{\tt #1}}}}



\name{Crash Intro to Git}

This is a short document on how to use the Git version control
system.\footnote{last modified October 16, 2016; based on ``Crash Intro to SVN''} It will teach you the
bare minimum you need to know to use it with game development.

\paragraph*{What is Git?}

Git is a ``version control system'': basically,
it tries to solve all of the things that can go wrong when multiple
people are editing, adding and deleting files from a big directory all
at once (two people trying to edit the same file, nobody knowing who
has the newest version of a given file, etc.). This is what you will
be doing while writing a game.

Although Git is a ``distributed VCS'', you will most likely want to use it in a centralized fashion, as follows:


\begin{itemz}[]

\item You will maintain one central repository with the latest version of your sheets. Most likely, this will be on \href{https://github.com}{Github} (a hosting service), but it could also be on \href{https://bitbucket.org/}{BitBucket} (another hosting service) or in an Athena locker.

\item Each user will have their own ``clone'' of the repository, which will also have all the sheets. You will write the game by modifying this local copy and ``pushing'' those changes back to the repository.

\item When a user asks for the most up-to-date version of the project, Git will give it to them, being careful not to overwrite changes that they've made on their copy.

\end{itemz}

We will talk about how to set it up and how to use it. First, make sure you have it installed: run \texttt{git --version} at a command prompt. On an Athena machine, or any other machine with git installed, this should return \texttt{git version \textit{...}}. If you don't have git installed and are running Debian or Ubuntu, install it with \texttt{apt-get install git}. On other systems, see \TODO{install instructions} for advice.

\section{Creating and populating a git repository}

{\em You will only have to do this step once per game.}

\subsection{Choosing a repo host}

You have several options for where to host your repository, with different advantages and disadvantages for each. You will most likely want a host that supports ``private'' repositories (ones only viewable by your GM team), so that your players don't accidentally spoil themselves.

\paragraph*{Github} Github is probably the biggest commercial Git host, and provides free public repositories to everyone. Unfortunately, private repositories generally require paying money, but if one of your GMs is a \emph{student}, they can get free private repositories through the \TODO{name of student pack}. If one of your GMs already pays Github for private repos, they can also create another one for free.

Github comes with a nice web interface, web-based collaboration tools (of somewhat dubious value for game-writing), and can be integrated with CircleCI\footnote{CircleCI is a ``continuous integration'' tool, which can help you discover mistakes you made in your \gametex{} before another GM runs into it. However, if nobody on your team is confident with the terminal and git, it's probably more complicated than its worth. See ``\gCI{}'' for details. \TODO{link to CircleCI docs}}.

If you choose to use Github, the GM with private repos available should go to \url{https://github.com/new}, name the repo after your game, mark it as private, and not create a README. After the repo has been created, go to Settings->Collaborators and add all your co-GMs to the list of collaborators.

\paragraph*{Bitbucket} Bitbucket is another large commercial Git host, which provides free public repos and \href{https://bitbucket.org/product/pricing/}{free private repos with up to five collaborators each}. Bitbucket has many of the same features as Github, including CircleCI support, but it's less widely used.

\paragraph*{\href{https://github.mit.edu}{github.mit.edu}} Github also has an ``enterprise'' version that MIT has purchased. This has much the same web interface as the normal Github, has unlimited free private repos, and uses Touchstone for authentication (which is convenient for people with Athena account and usable for those without), but doesn't support CircleCI. If you choose to use github.mit.edu, any GM can go to \url{https://github.mit.edu/new}, after which the procedure should be similar to the normal Github procedure.

\paragraph*{Athena} You can also host your git repo out of a GM's Athena locker. This loses the web interface of the other systems and requires an Athena account for any collaborators.

You will need an MIT mailing list for your team.  It should be a moira list (\emph{not} mailman) and an AFS group.

\TODO{Athena directions}

One of you should agree to host it in your locker, say in a directory called \texttt{gitgamename}. From the command line, you should do several things.

First, create the directory and give your GM team access to it (let's
say that your GMs are all on a list called \texttt{gamename-gms}. Then,
remove everybody else's ``list'' permissions from it. We'll call the
user creating the repo \texttt{joeuser}.

\ter{\$ mkdir gitgamename\\
\$ fs sa gitgamename system:gamename-gms write\\
\$ fs sa gitgamename system:anyuser none}

You can double check the permissions. They should look something like
this:

\ter{\$ fs la gitgamename\\
Access list for gitgamename is\\
Normal rights:\\
  system:gamename-gms rlidwk\\
  system:expunge ld\\
  joeuser rlidwka}

Then, let's create a fresh, blank repository inside:

\ter{\$ git init --bare gitgamename.git}

If you look inside the repo, it will have all sorts of mysterious directories.

\ter{\$ ls gitgamename.git/\\
branches/  config  description  HEAD  hooks/  info/  objects/  refs/}

\textbf{Do not manually change anything inside the repository}, unless you know how to use git (or are following directions from somebody who does). It will look like garbage anyway. Only use the Git commands we'll talk about in a second.

\subsection{Add \gametex{} to a repo}

Before you do this step, you should have a GameTeX tree already set up
in one of your lockers (see the ``Crash Intro to GameTeX'' greensheet;
life will be easier if you've already run Extras/changeclass.pl). The

This blank repository needs a copy of your GameTeX tree in it so that
you can start using the repo. To do this, you need to \emph{import} the
tree so that the repo can start keeping track of it. Go to the root of
a GameTeX tree you've previously set up, and run the following
command. A text editor will pop up. Just enter ``Initial commit'' and
exit the text editor.

\ter{GameName\% svn import . file:///mit/joeuser/svngamename\\
Adding         README\\
Adding         Bluesheets\\
Adding         Bluesheets/README.tex\\
... lots of other output ... \\
Committed revision 1.}

Now your repo contains a copy of your GameTeX tree and is ready to be
used.

\paragraph*{Checking out a copy}

\emph{You will only have to do this step once per GM.}

Good. Now that the repository is up and running in {\tt joeuser}'s
locker, each GM needs to check out a local copy of it. Since the repo
is running out of {\tt joeuser}'s locker, let's make sure it is always
attached by adding the following line to your {\tt .environment}:

\ter{add joeuser}

Now, create the folder to wherever you decided your local copy will
sit (you probably want it somewhere like\\ {\tt
/mit/yourusername/GameName/}, but consult the ``Intro to GameTeX''
greensheet). The SVN command to checkout a copy from the repo has this
form:

\ter{\% svn checkout file:///mit/joeuser/svngamename GameName/\\
A    GameName/README\\
A    GameName/Bluesheets\\
A    GameName/Bluesheets/README.tex\\
... lots of other output ... \\Checked out revision 2.}

Above, make sure to replace {\tt /mit/joeuser/svngamename} with the
actual location of the repository up above and {\tt GameName} with the
directory you want to keep your game tree in. After all is said and
done, you will now have a working local copy of the repository. Each
GM should do this, including the one who created the repository. Your
checked out copy will have hidden {\tt .svn/} folders all over the
place that you may notice. Don't mess with those - that's just how SVN
keeps track of some information.

\paragraph*{Updating and committing}

This section will tell you about the most common operations. Once you
have checked out a copy of the repository, you can start to use SVN in
full. To update your local copy with the newest version of the
repository, use the {\tt svn update} command.

\ter{\% svn update\\
U  Greensheets/magic.tex\\
G  Bluesheets/anarchists.tex}

A ``U'' means that a file was updated with the newest version. A ``G''
means that the file was updated with the newest version, but you had
some local changes in the file, and SVN merged the two just fine.  A
``C'' means there was a conflict, but we'll talk about that in the
next section. An ``A'' means that the file was recently added to the
repo and that this is your first time you are receiving a copy. A
``D'' means the file was deleted. {\bf You should update early and
often.}

After you have made some changes to a file, you should {\em commit}
them to the central repository, so that when everybody else updates,
they get a copy of your changes. When you commit, you will be asked to
supply a {\em log}, which is any written comments you want to make for
your other users. You can also just specify the log message with {\tt
-m}.

\ter{\% svn commit Lists/char-LIST.tex -m "Gave James Bond a pistol."\\
Sending     Lists/char-LIST.tex\\
Transmitting file data.\\
Committed revision 43.}

Now, whenever the other users update, their copy of {\tt
char-LIST.tex} will include your changes. You can commit individual
files, or entire directories. {\bf You should commit early and often.}

There's one more issue. If a version of the file already exists in the
repository, you can update and commit normally. However, if you want
to add a new file for the first time, you have to use the {\tt add}
command. This will schedule the file for inclusion next time you
commit:

\ter{\% svn add Charsheets/nickfury.tex\\
A         Charsheets/nickfury.tex\\
\% svn commit Charsheets/ -m "Added Nick's charsheet."\\
Sending     Charsheets/nickfury.tex\\
Transmitting file data.\\
Committed revision 46.}

Similarly, you can {\tt add} individual files or entire directories
all at once. When other users update, they will receive a copy of {\tt
nickfury.tex} for the first time. Then you can all update and commit
changes to it normally.


\paragraph*{Handling conflicts}

When Alice and Bob are make changes to the same file, Alice will
commit her changes and Bob will have to update his local copy before
commiting his changes. Most of the time, SVN will merge Alice's
changes intelligently into Bob's local copy and integrate their
changes (and Bob will see a ``G'' when he updates). Sometimes, for
whatever reason\footnote{Most likely, that Alice and Bob and were
making changes to the exact same place.}, SVN doesn't trust itself
that it knows how Alice and Bob want their changes merged. When this
happens, Bob will see a ``C'' when updating, which stands for {\em
conflict}.

\ter{\% svn update\\
C Charsheets/warlock.tex}

SVN will markup any file that is in {\em conflict} with special
symbols to show you where the conflict is, showing Bob both Alice's
version that is now in the repo and Bob's version, and where they
disagree. Bob will have to manually merge the changes. When Bob looks
at the file, he might see a lot of blocks like this:

\ter{<<<<<<< .mine\\
and so you came to Moscow to kill your enemy James Bond.\\
=======\\
and so you came to Moscow to help your friend Nick Fury.\\
>>>>>>> .r54}

Every block between the less than and greater than signs is a part of
the file that is in conflict. What Bob's local file has is shown after
the {\tt <<<<<<< .mine}. What the repository has (in this case,
Alice's changes) is shown above the {\tt >>>>>>> .r54}.\footnote{The
``54'' is just indicating tha the current revision of the project in
the repo is revision \#54.} The {\tt =======} deliniates between the
two versions. Bob needs to get rid of this whole block and merge the
two changes so that they make sense:

\ter{and so you came to Moscow to kill your enemy James Bond and help your friend Nick Fury.}

When Bob is done editing the file so that it merges Alice's and his
changes the way they want, he needs to tell SVN that he fixed the
problem with the {\em resolved} command.

\ter{\% svn resolved Charsheets/warlock.tex \\
 Resolved conflicted status of Charsheets/warlock.tex !}

Now Bob can commit normally. When Alice updates, SVN will give her the
correct copy of {\tt warlock.tex} with both sets of changes.

\paragraph*{Afterword}

SVN is one of those things that you won't be able to imagine you lived
without. It will help get your whole house in order. Commit and update
early and often, and you will save yourself and your GM team a lot of
pain. Type {\tt svn help} for more documentation, or look it up
online.

\end{document}
